\documentstyle[11pt]{article}
\flushbottom
\addtolength{\topmargin}{-.8in}
\addtolength{\oddsidemargin}{-.9in}
%\addtolength{\oddsidemargin}{-.7in}
\addtolength{\textheight}{1.6 in}
\addtolength{\textwidth}{1.6 in}
%\addtolength{\textwidth}{1.4 in}
\renewcommand{\baselinestretch}{1.05}
\pagestyle{empty}
\thispagestyle{empty}
\title{\large {\bf Semester Project -- Part 3 -- CSE4100 -- Fall 2011}}
\author{\large {\bf Due: Wednesday, December 7, 2011, at 2:00pm}}
\pagestyle{empty}
\thispagestyle{empty}
\begin{document}
\date{}
\pagestyle{empty}
\thispagestyle{empty}
\maketitle
%\footnotetext{\rule{0pt}{0.75in}}
%\addtolength{\textheight}{.25 in}
\normalsize
\vspace{-0.13in}

The third and final part of the semester project focuses on the actual
generation of formatted output from Latex input.  There are 5 tasks
for this project, with the divisions given to differentiate between 
required work/point totals and bonuses.  Note that you {\bf MUST} utilize the
{\tt \%union} command to redefine the parsing stack in bison (or an 
equivalent redefinition in whichever bison you are using).
This project is worth 150 points.   The tasks for the project are:
\begin{enumerate}
\item Basic Text Processing Capabilities (35 points total),
      including: Section/Subsection/Table of Contents (5pts),
      Line Spacing/Single-Double-Triple (5pts), Page Numbering/Styles (2.5pts),
      Horizontal/Vertical Spacing (2.5pts), Italics/Roman Fonts (2.5pts),
      Paragraphs/Noindent (2.5pts), Right Justification (10pts), and
      Begin/End Single Blocks (5pts).
\item Advanced Text Processing capabilities (55 points total),
      including: Itemize Blocks (5pts), Enumerate Blocks (5pts),
      Center Blocks (5pts), Verbatim Blocks (5pts), Tabular Blocks (10pts),
      Table Blocks with Refs/Captions (5pts), and
      Relevant Combinations of Blocks (20pts).
\item Nested Blocks within Single Environment (15 points total): In this task, you must
      rewrite the grammar to support the definition of single blocks with 
      nested blocks that have intervening text, e.g.,
      \begin{small}
      \begin{verbatim}
      \begin{single}
      Here is some text before 
      a begin end itemize block.

      \begin{itemize}
      \item One item
      \item Another item
      \end{itemize}
      Here is some more text - does this work?
      \end{single}
      \end{verbatim}
      \end{small}
      To support this capability, you must carefully alter the grammar
      production rules.  We recommend doing this last, since it will likely
      cause problems and the other parts can be designed, implemented,
      and tested, without this extension.
\item Full-Blown Verbatim (15 points total): All possible Latex commands that are 
      embedded in a verbatim block (e.g., backslash, blocks, etc.) are ignored.
      The single block previously generated was output by enclosing all
      of the commands within a verbatim block.
\item Type/Error Checking (20 points total): Add type checking and error reporting to
      your Latex compiler.  Implement the following type checking:
      \begin{itemize}
      \item Basic Begin/End Blocks (5pts): the type of block for the begin
            and end must match, e.g.,  both single, both verbatim, etc.
      \item Adv. Begin/End Blocks (5pts): Only supports the following combinations:
            \begin{small}
            \begin{verbatim}
                Single around Itemize/Enumerate/Center
                All Combos of Itemize/Enumerate - Up to 3 levels deep
                Center around Tabular/Verbatim
            \end{verbatim}
            \end{small}
            All other combinations should result in an error.
      \item Tabular Specification (10pts):  The number of columns in the table
            specification should match each table entry.  For example:
            \begin{small}
            \begin{verbatim}
                 \begin{tabular}{ccc} 
                 One & Two & Three \\
                 One & Two & Three & Four \\
                 One & Two \\
                 \end{tabular}
            \end{verbatim}
            \end{small}
            has an error in the second entry since four columns are given.  
            Note that the third entry is allowed since there are less
            columns than specified, so Latex assumes a blank third 
            column for the entry.
      \end{itemize}
\item Documentation, Log, Testing (10 points total).  Note that your 
      documentation {\bf MUST} be run through your 
      text processor to generate formatted output from a Latex input file!
\end{enumerate}
\noindent
Note that you can have multiple versions of your ``completed'' project.
For example, you may have a version that supports Basic and Advanced Text 
Processing Capabilities and Type-Checking, with a second version containing
Verbatim and Nested Blocks enhancements.  Points for each task are posted
to allow you to maximize the points you receive for the project.

The course web site contains a number of important files, including:

\begin{small}
\begin{verbatim}

WEB PAGE FILES FOR PROJECT, PART 3

latex.l          : Common lexical analyzer specification
latexp3c.y       : Yacc file with nested blocks, WS, and verbatim
                   along with basic code generation
latexp3c.output  : S/R and R/R Conflicts - Are all OK?

generate.c       : Basic routines for formatted text generation
util.c           : Utility routines

latex.input.txt  : Sample input
latexout.txt     : Generated output for sample (with errors!)
latextoc.txt     : Generated table of contents for sample

proj3gs.doc      : Grading Sheet - place initials next to which parts
                   each person on the team was primarily responsible for.




COMPILATION INSTRUCTIONS:

flex latex.l
bison latexp3c.y
gcc latexp3c.tab.c -lfl

EXECUTION INSTRUCTIONS:

a.out < latex.input.txt

THIS GENERATES THE FILES:

latexout
latextoc
\end{verbatim}
\end{small}

\noindent
Portions of these and other files will be discussed in class.  
Note that the files latex.input.txt, latexout.txt, and latextoc.txt
all have ``.txt'' extensions for viewing on the course website.  However,
the code (latex.l, latexp3code.y, util.c, and generate.c) all use
these files without the ``.txt'' extension.

The third part of the project is a {\bf TEAM PROJECT} and is due on Wednesday, 
December 7, 2011 at 2pm.  You are to work in teams of 2 individuals.  
You must notify Prof. Demurjian (steve\@engr.uconn.edu) of your team by November 11th.  
For the final project report, please hand in the following:
\begin{enumerate}
\item A hard copy of the bison specification for part 3.  DO NOT HAND 
      IN the generated bison {\tt ``.c''} files!  
      Also include copies of any relevant {\tt ``.c''} files! 
\item Documentation of your solution that includes: (1) any assumptions that you
      make regarding the output format and style of your document (e.g., margins,
      page number locations, handling underlining, etc.);  (2) a log file that 
      keeps track of all of your major design steps, implementation strategies, 
      problems encountered (with flex/bison) and their solutions, aspects of the 
      project that were easy/hard, changes that were made to the grammar, etc.  
      The key word in (2) is {\bf MAJOR} design steps.  NOTE AGAIN THAT
      YOUR DOCUMENTATION MUST BE GENERATED USING YOUR LATEX COMPILER!!!!
      (Thus, hand in both the documentation and the original input file!)
\item Test cases and test results for all tasks, clearly marked and organized.
      Note that in late November/early December, a set of representative test
      cases may be made available for your use.
\item Hand in your projects using an electronic media (via email - zip file using lastnames.zip)
      emailed to TA Eugene Sanzi({\tt cse4100projects@gmail.com}). 
\end{enumerate}
\noindent
 Note that if Eugene has trouble compiling and/or executing your software, 
he will contact you for an in person demo.`
\end{document}
